\chapter{Opinnäytetyön rakenne}

Tässä esimerkkejä luvuista ja aliluvuista.

\section{Matemaattisista merkinnöistä}

\LaTeX:lla yhtälöiden kirjoittaminen on helppoa. Seuraavassa ensin esimerkki 

\begin{equation}
  \label{eq:fourier}
  G^+(t,t')= \int G^+(E) exp[-iE(t-t')/\hbar] dE,
\end{equation}

jonka jälkeen kappale jatkuu näin.
Tekstissä kaavat $G^+(t,t')= \int G^+(E) exp[-iE(t-t')/\hbar] dE$ näyttävät tällaisilta. 
Kaavoihin (ks. \ref{eq:fourier}) on myös helppo viitata.

% Esimerkkikuva.
% Huomaa, että useimmat ohjelmat latovat PDF-kuvan
% A4-sivulle, jolloin kuvassa on mukana paljon tyhjää.
% Tämän voi poistaa PDF-editoreilla. Esim.
% PDF-X-Change-ohjelmistossa on toiminto "remove white spaces".
%
% Huomaa myös, että Latex sijoittaa kuvan aikaisintaan
% tähän kohtaan tekstissä. Yleensä kuvat kannattaa
% ladata n. sivun verran aikaisemmin kun niihin viitataan.
% Muuten seuraava sopiva paikka saattaa tulla vasta
% parin sivun päähän viittauksesta.

\centeredpicture{esimKuva}{Matlabilla tehty PDF-muotoinen esimerkkikuva.}

Latexiin voi tuoda kuvia useassa eri formaatissa.
Näppärintä on käyttää pdflatex-kääntäjää, jolloin sallitut formaatit ovat \verb+*.png+, \verb+*.jpg+ sekä \verb+*.pdf+.
Kuvassa~\ref{fig:esimKuva} on PDF-formaatissa tuotu Matlabilla tehty esimerkkikuva.

Tässä luvussa esitellään opinnäytetyössä mahdollisesti esiintyviä dokumentin ominaisuuksia ja kuinka niitä voidaan käyttää \LaTeX:n avulla. 

\section{Viittaustekniikka}

Opinnäytetyössä käytetään BibLaTeX:n authoryear-tyyliä, joka on muokattu vastaamaan TTY:n opinnäytetyöohjetta.
Seuraavassa viittausesimerkkejä:

\begin{itemize}
	\item \textcite[ss.~32--34]{lehman2014biblatex} mukaan ...
	\item Many ERP implementations have been classified as failures because they did not achieve predetermined corporate goals \cite{umble2003enterprise}.
	\item ... sivunumerot ja monta lähdettä \cites[189]{osmani2013}[1]{knuth1973fundamental}.
	\item Viittaus voi olla myös usean virkkeen mittainen. Tällöin käytetään \verb+\parencite*+-komentoa. \cite*{somers2001impact}
	\item Suomenkielisissä teoksissa tekijän nimeä tulee toisinaan taivuttaa. Osmanin \citeyear[12]{osmani2013} mukaan ...
\end{itemize}

Lähdeluettelo koostetaan automaattisesti tekstissä esiintyvien viitteiden mukaan.
Tästä johtuen dokumentin kääntäminen pdf-tiedostoksi vaatii useamman \LaTeX-kääntäjän ajon.
Dokumentti kannattaakin kääntää komennolla \texttt{pdflatex biber pdflatex pdflatex}, jotta sekä viittaukset että lähdeluettelo päivittyvät oikein.

Lähdeluettelon esitystapa on kuvattu taulukossa \ref{table:bibliography}.

\begin{table}
\caption{Lähteiden merkintätapa lähdeluettelossa}
\label{table:bibliography}
\begin{center}
\begin{tabular}{ | p{7cm} | p{7cm} | }
	\hline
	\textbf{Tiedot ja niiden järjestys} & \textbf{Esimerkki} \\ \hline
	Kirja \newline 	Tekijät ja julkaisuvuosi. Otsikko. (Painos, jos useita.), Julkaisupaikka, Julkaisija. Sivumäärä. & \fullcite{basar1995dynamic}. \\ \hline
	Artikkeli \newline Tekijät ja julkaisuvuosi. Otsikko. Lehden nimi. Vol. x(nro), sivut. & \fullcite{umble2003enterprise}. \\ \hline
	Konferenssiesitelmät \newline Tekijät ja julkaisuvuosi. Otsikko. Konferenssin nimi, Paikka, Aika, Julkaisupaikka, Julkaisija, sivut. & \fullcite{somers2001impact}. \\ \hline 
	Artikkeli kokoelmateoksessa \newline Tekijät ja julkaisuvuosi. Otsikko. Teoksen toimittajat. Teoksen nimi. (Painos, jos useita). Julkaisupaikka, Julkaisija, sivut. & \fullcite{osmani2013}. \\ \hline
	Verkkojulkaisu \newline Tekijä. Julkaisuvuosi. Julkaisun nimi. URL-osoite. Viittauksen päiväys. & \fullcite{lehman2014biblatex}. \\
	\hline
\end{tabular}
\end{center}
\end{table}

