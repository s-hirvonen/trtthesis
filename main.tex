\documentclass[12pt,a4paper,finnish]{vendor/tutthesis}
\graphicspath{ {./vendor/images/} {./images} }

% LaTeX-tiedosto opinnäytepohjaksi
% Tekijä: Sami Paavilainen
% Muokkaaja: Heikki Huttunen (heikki.huttunen@tut.fi) 31.7.2012.

% Vaatii lisäksi luokkatiedoston tutthesis.cls
% sekä tiedoston tty-logo.pdf (pdflatex) tai tty-logo-eps (latex)

\usepackage[finnish]{babel} % tavutus suomeksi, vaihda jos eri kieli
\usepackage[small]{caption}% kuvatekstien kirjasinkoko ja asettelu
%omia käskyjä voi lisätä tähän väliin, esim. \newcommand{\angs}{\textsl{\AA}}
\pagenumbering{Roman}
% alkuvalmistelut loppuu

\begin{document}

% Comment these lines out as needed
\makeatletter
\let\thetitle\@title
\let\theauthor\@author
\makeatother

\thispagestyle{empty}

\vspace*{-.5cm}\noindent

% If thesis is in English, use the file "tut-logo"
% instead of "tty-logo" in the following:

\includegraphics[width=8cm]{tty-logo}

\vspace{6.8cm}

\MakeUppercase{{\large\textbf{\textsf{%
\uppercase{\theauthor} \\
\thetitle \\
}}}}
\textsf{\documenttype}

\vspace{8.7cm} % jos kaksi otsikkoriviä vaihda -> 6.7cm

\begin{flushright}
\begin{minipage}[c]{6.8cm}
\begin{spacing}{1.0}
\begin{flushleft}
\textsf{\nohyphens{Tarkastaja: \inspector\ Tarkastaja ja aihe hyväksytty \approvaldate.}}
\end{flushleft}
\end{spacing}
\end{minipage}
\end{flushright}

\cleardoublepage

\nnchapter{Tiivistelmä}

\begin{spacing}{1.0}
\textsf{%
\MakeUppercase{Tampereen teknillinen yliopisto} \\
xxxxxxxxxxxxx koulutusohjelma \\
\textbf{\MakeUppercase{\theauthor}: \thetitle} \\
\documenttype, \getpagerefnumber{LastPage} sivua \\
\monthyear \\
Pääaine: \\
Tarkastajat: \\
Avainsanat: }	
\end{spacing}

Vivamus nec ullamcorper felis, sed egestas dui. 
Suspendisse condimentum metus dapibus, viverra mauris vel, malesuada felis. 
Praesent imperdiet orci a elit aliquet hendrerit. Nullam et euismod enim. 
Etiam euismod tellus sed est condimentum adipiscing. 
Suspendisse vel tortor quis nisi fermentum porta. 
Ut quis tempus eros, et aliquam ligula. 
Quisque nec nisl libero; pellentesque euismod nulla id risus egestas, vitae convallis eros vestibulum. 
Praesent aliquet fringilla sagittis.

Nunc lobortis faucibus sollicitudin. 
Fusce semper pulvinar nulla ac posuere. 
Ut non sapien elementum, blandit dolor et, aliquam erat. 
Aliquam feugiat est dui, sed accumsan turpis dictum non. 
Nullam scelerisque id ipsum vel sodales. 
Nullam id iaculis mauris, pretium vestibulum enim.
Phasellus ligula risus, feugiat suscipit consectetur quis, varius vitae justo. 
Etiam aliquet aliquet risus, a molestie nisl placerat vitae. 
Donec at pulvinar mi. Vivamus viverra volutpat purus id elementum. 
Sed a elementum nisi. 
Maecenas eu semper tortor, nec congue ligula. 
Sed eget lectus facilisis, hendrerit nibh a, ornare libero. 
Etiam elementum leo sit amet velit fringilla aliquam.

\chapter*{ABSTRACT}
\begin{spacing}{1.0}
\textsf{TAMPERE UNIVERSITY OF TECHNOLOGY}\\
\textsf{Master's Degree Programme in xxxxxxx Technology}\\
{\bf \textsf{AUTHOR : Title}}\\
\textsf{Master of Science Thesis, xx pages, x Appendix pages}\\
\textsf{xxxxxx 201x}\\
\textsf{Major: }\\
\textsf{Examiner: }\\
\textsf{Keywords: }\\
\end{spacing}

\noindent
First paragraph

\noindent
Second paragraph
\nnchapter{Alkusanat}

\noindent Tämä (\textit{d-tyo.tex}) on \LaTeX-pohja Tampereen teknillisen
yliopiston opinnäytetöitä varten. Samaan pakettiin kuuluu myös
tiedosto \mbox{\textit{tutthesis.cls}}, joka sisältää taittoteknisiä
lisäyksiä \LaTeX:n alkuperäiseen \textit{report.cls}-luokkatiedostoon.

Lisäksi otsikkosivua varten tarvitaan tiedosto \textit{tty-logo.xxx}, jonka
tulee sisältää TTY:n logo. Tiedoston tulee
olla joko \textit{.eps}- tai \textit{.pdf}-muodossa riippuen \LaTeX-versiosta.
\renewcommand\contentsname{Sisällys} % Sisällysluettelon nimi
\tableofcontents
%\listoffigures
%\listoftables
\nnchapter{Lyhtenteet ja merkinnät}

\begin{termiluettelo}

\item [$\hbar$] Redusoitu Planckin vakio
\item [SNR] Signaali-kohinasuhde (engl.: Signal to Noise Ratio)

\end{termiluettelo}

\renewcommand{\chaptermark}[1]{\markboth{\thechapter. \ #1}{}}
\renewcommand{\sectionmark}[1]{\markright{}{}}
\lhead{\fancyplain{}{\leftmark}}

% Tästä alkaa varsinainen teksti
\pagenumbering{arabic}

\chapter{Johdanto}

Tampereen teknillisen yliopiston perustutkintojen opinnäytetöitä ovat
kandidaatintyö ja diplomityö. Lisäksi tehdään kirjallisia
harjoitustöitä, joissa tätä ohjetta voidaan käyttää soveltuvin osin.

Tämä pohja on tarkoitettu auttamaan työn kirjoittamista \LaTeX:n
kanssa. Oppaita itse \LaTeX:n käyttöön löytyy kattavasti verkosta,
esimerkiksi Wikipediasta. Myös seuraava opas on hyvä lähtökohta
aloittelijoille:
\begin{quote}
http://www.ctan.org/tex-archive/info/lshort/english/lshort.pdf
\end{quote}

\chapter{Opinnäytetyön rakenne}

Tässä esimerkkejä luvuista ja aliluvuista.

\section{Lähdeluettelo}

Lähdeluettelo voi \LaTeX:ssa tehdä joko tietokantamuotoisesti
\textit{bibtex}-käskyn avulla tai vaihtoehtoisesti tässä esitetyllä tyylillä.
Lisätietoa löytyy esim. Wikipediasta ''\textit{bibtex}''-hakusanalla.

Tältä näyttää viittaus perinteisellä numeromerkinnällä \cite{Hirs,Mittelbach}.

% Huom.: oikeaoppinen tapa viitata olisi seuraava:
%
% ...numeromerkinnällä~\cite{Hirs}.
%
% Merkintä "~" toimii kuten välilyönti, mutta estää rivinvaihdon
% ja seuraavan rivin alkamisen rumasti viittauksella.

\subsection{Aliesimerkki}

Turhaa tekstiä esimerkin vuoksi.

\section{Matemaattisista merkinnöistä}

\LaTeX:lla yhtälöiden kirjoittaminen on helppoa. Seuraavassa ensin esimerkki
\begin{equation}
  \label{eq:fourier}
  G^+(t,t')= \int G^+(E) exp[-iE(t-t')/\hbar] dE,
\end{equation}
jonka jälkeen kappale jatkuu näin.
Tekstissä kaavat $G^+(t,t')= \int G^+(E) exp[-iE(t-t')/\hbar] dE$ näyttävät
tällaisilta. Kaavoihin (ks. \ref{eq:fourier}) on myös helppo viitata.

% Esimerkkikuva.
% Huomaa, että useimmat ohjelmat latovat PDF-kuvan
% A4-sivulle, jolloin kuvassa on mukana paljon tyhjää.
% Tämän voi poistaa PDF-editoreilla. Esim.
% PDF-X-Change-ohjelmistossa on toiminto "remove white spaces".
%
% Huomaa myös, että Latex sijoittaa kuvan aikaisintaan
% tähän kohtaan tekstissä. Yleensä kuvat kannattaa
% ladata n. sivun verran aikaisemmin kun niihin viitataan.
% Muuten seuraava sopiva paikka saattaa tulla vasta
% parin sivun päähän viittauksesta.

\begin{figure}[t]
\begin{center}
\end{center}
\caption{Matlabilla tehty PDF-muotoinen esimerkkikuva.}
\label{fig:esimKuva}
\end{figure}

\section{Kuvat}

Latexiin voi tuoda kuvia useassa eri formaatissa.
Näppärintä on käyttää pdflatex-kääntäjää, jolloin
sallitut formaatit ovat \verb+*.png+, \verb+*.jpg+
sekä \verb+*.pdf+. Kuvassa~\ref{fig:esimKuva} on
PDF-formaatissa tuotu Matlabilla tehty
esimerkkikuva.

% lähdeluettelo
\newpage
\addcontentsline{toc}{chapter}{Lähteet} % Lisää tämäkin sisällysluetteloon
\renewcommand{\bibname}{Lähteet}
\begin{thebibliography}{99}
\bibitem{Hirs} Hirsjärvi, S., Remes, P., ja Sajavaara, P. 2005. Tutki ja
  kirjoita, 11. painos. Helsinki, Tammi. 436 s.
\bibitem{Mittelbach} Mittelbach, F., Goossens, M., Braams, J.,
Carlisle, D., Rowley, C. 2004. The Latex Companion, 2. painos. Boston, Addison-Wesley. 1120 s.

\end{thebibliography}

% seuraavia voi käyttää bibtex:n kanssa (edellyttää tiedoston kirjat.bib)
%\bibliographystyle{plain}
%\bibliography{kirjat}

\appendix
\renewcommand{\chaptername}{Liite}

\chapter{Liitteitä}
\end{document}

